\documentclass{article}
\usepackage[utf8]{inputenc}
\title{Algorithmic trading}
\author{lkathary@student.21-school.ru, rglorfin@student.21-school.ru}
\date{March 2023}

\begin{document}
\maketitle

\section{Algorithmic trading}
\begin{itemize}
    \item The program was developed in C++ language of \textbf{C++17} standard using g++ (GNU C++) compiler.
    \item GUI implementation, based on the \textbf{QT} library with API for \textbf{C++17}.
    \item The program provides the ability to:
        \begin{itemize}
            \item Plot a tabulated function of stock quotes using two interpolation methods.
            \begin{itemize}
            \item Newton polynomial of \(n\)th degree (\(1 \leq n \leq 9\)).
            \item Cubic Spline.
        \end{itemize}
    \item Plot an approximation (by a polynomial of \(m\)th degree (\(1 \leq m \leq 20\)))
           of a tabulated function of stock quotes using the least squares method.
           The approximation can also be extended to the future to predict the stock price.
    \end{itemize}
\end{itemize}

\begin{itemize}
    \item The program GUI contains:
    \begin{itemize}
        \item Two tabs: Interpolation and Approximation, each with a drawing area.
        \item Data manipulation area with buttons to do the following:
            \begin{itemize}
                \item Load data (stock quotes) from a CSV-file.
                \item Clear the data.
                \item Reset the drawing area (only up to 5 graphs can be shown simultaneously).
                \item Show some information about the data.
                \item Clear the area with that information.
            \end{itemize}
    \end{itemize}
\end{itemize}

\end{document}
